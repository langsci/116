\addchap{Vorwort}
% \begin{refsection}

%\epigraph{\textit{A Huhn, wos kräjt, un a Goi, wos schmusst jiddisch – sollen sein Kapore far mir.}}{--- \textup{Jiddisches Sprichwort (\citealt[524]{Landmann2013})}} 

\noindent Die vorliegende Untersuchung zu Imitationen des Jiddischen in der deutschsprachigen Literatur entstand zwischen 2012 und 2014 als Dissertationsschrift an der Philipps-Universität Marburg. Die literarische Darstellung, Funktionalisierung und Struktur von (jiddischer) Mündlichkeit beschäftigt mich seit meinem Grundstudium. Aber erst im Rahmen des \hai{DFG}-Projekts \qu{Westjiddisch im (langen) 19. Jahrhundert: Quellenlage, soziolinguistische Situation} (2011--2016 Universität Marburg) wurde mir bewusst, welche Popularität die markierte Figurenrede jüdischer Charaktere in der deutschsprachigen Literatur des 18. und 19. Jahrhunderts genoss und wie diese literarische Tradition die sprachliche Darstellung von Juden im deutschen Sprachraum bis heute beeinflusst. Im \hai{DFG}-Projekt konnte eine Vielzahl literarischer Texte nicht-jüdischer Autoren erschlossen werden, in denen jüdische Figuren über eine von der Norm abweichende Sprache charakterisiert werden. Vor dem Hintergrund, dass dieser Quelltyp nicht unserem \qu{Beuteschema} authentischer jiddischer Quellen entspricht und wir nicht gezielt nach solchen Texten nicht-jüdischer Autoren suchten, ist unserer Einschätzung nach die Dunkelziffer von solchen Publikationen in diesem Zeitraum deutlich höher, als unser Projektsample abbildet. Diese Spitze des Eisbergs bildet die Datengrundlage der vorliegenden Untersuchung. 
Es ist meinen Doktoreltern Jürg Fleischer und Marion Aptroot zu verdanken, dass Sie mir alle Freiheiten eingeräumt haben, diese anfangs nicht besonders vielversprechende Untersuchung von vorwiegend antisemitischen Texten durchzuführen. 

Größere Kürzungen gegenüber dem Manuskript (\citealt{SchaeferDiss}) 
%linktomanuscript verweis auf digitale version der urpsrünglichen Arbeit
wurden in der vorliegenden Publikation in der Darstellung des Projektsamples und im Appendix vorgenommen. Auch wurde auf den gesamten Datenbereich zu Quellen aus dem 21. Jahrhundert  verzichtet, da so eine Konzentrierung auf das (lange) 19. Jahrhundert gewährleistet ist. 
%$\triangleq$

Ohne die Hilfe und den Austausch mit Kollegen wäre diese Arbeit in der vorliegenden Form nicht möglich gewesen. Ein nicht unwesentlicher Teil des Projektsamples ist dem Spürsinn Ute Simeons zu verdanken. Über das Projekt hinaus haben verschiedene Menschen und Maschinen zum Entstehen dieser Arbeit beigetragen. Ohne die frei zugänglichen Digitalisate von Bibliotheken und Online-Dienstleistern, allen voran 
\href{https://archive.org}{Archive.org}
und 
\href{books.google.de}{Google Books}
hätte das der empirischen Analyse zugrundeliegende Korpus nicht in dieser Form entstehen können. Allein in Anbetracht der Datengrundlage ist es m.\,E. sinnig den Open Access %rs +s, evtl. \textit{Open Access}-Gedanken
Gedanken zu leben. Es freut mich daher sehr, dass mir dies der Verlag \textit{Language Science Press} auf akademisch höchstem Niveau ermöglicht. Den Herausgebern der Reihe \qu{Language Variation}, insbesondere John Nerbonne, den Gutachtern und Proofreadern, danke ich für Ihre nützlichen und interessanten Hinweise. Sebastian Nordhoff und Felix Kopecky (Language Science Press) danke ich für den Support in Sachen Overleaf und die rundum Betreuung der vorliegenden Publikation. Danken möchte ich auch all jenen, die mit kleineren und größeren Anregungen, Hinweisen und Proofreadings diese Arbeit beeinflusst haben:
Magnus Breder Birkenes,
Sabine Boehlich (†),
Michael Cysouw,
Alexander Dröge,
Agnes Kim,
Stephanie Leser-Cronau,
Jürgen Lorenz,
Jeffrey Pheiff,
Alexis Manaster Ramer,
Jona Sassenhagen, 
Oliver Schallert,
Julia Schüler,
Jan Süselbeck und
Karin Weiss.
Im Hessischen Staatsarchiv Marburg wurden mir anregende Archivrecherchen ermöglicht. Beatrice Santorini hat mir mit der Bereitstellung ihres diachronen Korpus zum Jiddischen einen zentralen Vergleichsdatensatz geliefert. 
Mein besonderer Dank gilt Ricarda Scherschel für ihren tatkräftigen Beistand als studentische und später wissenschaftliche Hilfskraft im Projekt und für ihr intensives und gewissenhaftes Lektorat der vorliegenden Buchversion. Darüber hinaus stand sie mir jederzeit (und von jedem Ort der Welt aus) bei \TeX-nischen Fragen und Problemen zur Seite. 


Für alles über den fachlichen Beistand weit Hinausgehende danke ich meiner Familie, Sascha Peter, Alexander Dröge, Jona Sassenhagen, Katerina Danae Kandylaki, Phillip Alday, Lisa Martin, Jaruwan Junnum, sBopi und sRösile, allen Ukulelisten, Musikern und Sängern der \qu{Lingulelen}, \qu{Boogie Bats} und \qu{Wirsingquerbeet}. Der Landesbibliothek Vorarlberg danke ich für WLAN und die nötige Ruhe um Dinge voran zu bringen. Für Erdung~\mbox{\begin{tikzpicture}[circuit ee IEC] \draw (0.5,0) to (0.5,-0.18) to [ground](0.5,-0.25);\end{tikzpicture}}~danke ich Oli, mein \textit{basherter} (in \RL{הייא}), und Oma (im All).\\ 
%\noindent Lea Schäfer  \hspace{6.5cm} Bregenz,  Herbst 2016 \\ 

\hspace{9.5cm} Weihnukka 2016\\



%\noindent Lea Schäfer  \hspace{6.5cm} Bregenz,  Herbst 2016 \\ 

% \printbibliography[heading=subbibliography]
% \end{refsection}