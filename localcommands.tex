%add all your local new commands to this file

\newcommand{\smiley}{:)}

\clubpenalty=10000
\widowpenalty=10000

\newcommand{\farbgrafik}{\addtostream{colorfigures}{\thepage}} %für Abbildungen die in Farbe gedruckt werden sollen

\newcommand{\cmark}{\ding{51}}%%fettes \checkmark zeichen

\newcommand{\hai}[1]{\textrm{#1}} %textsf für Text in Arial
\newcommand{\qu}[1]{„#1“\xspace} % >> <<
\newcommand{\quji}[1]{„#1“\xspace} %jiddische (dt.)
%\newcommand{\qu}[1]{»#1«} % >> <<
%\newcommand{\quji}[1]{»#1«} %jiddische (dt.) Anführungszeichen
\newcommand{\quein}[1]{\glqq#1\grqq\xspace}%{»#1«} %{›#1‹} %einfache > <
\newcommand{\quf}[1]{\frqq#1\flqq} %franz. Anführungszeichen
\newcommand{\qufs}[1]{\frq#1\flq} %einfache franz. Anführungszeichen
\newcommand{\sem}[1]{‘#1'} % 69er anführungszeichen Bedeutungsangabe

\newcommand{\bs}{\ensuremath{\backslash}}

\newcommand*{\haifont}[1]{\textsf{#1}}

\newcolumntype{Y}{>{\centering\arraybackslash}X}

\definecolor{Fuchsia}{rgb}{1.0, 0.0, 1.0}
\definecolor{BrickRed}{rgb}{0.8, 0.25, 0.33}
\definecolor{MidnightBlue}{rgb}{0.1, 0.1, 0.44}
\definecolor{Thistle}{rgb}{0.85, 0.75, 0.85}
\definecolor{LimeGreen}{rgb}{0.2, 0.8, 0.2}
\definecolor{ProcessBlue}{rgb}{0.0, 0.72, 0.92}
\definecolor{Maroon}{rgb}{0.5, 0.0, 0.0}
\definecolor{Melon}{rgb}{0.99, 0.74, 0.71}
\definecolor{RoyalPurple}{rgb}{0.47, 0.32, 0.66}
\definecolor{YellowGreen}{rgb}{0.68, 1.0, 0.18}
\definecolor{CarnationPink}{rgb}{1.0, 0.65, 0.79}
\definecolor{SkyBlue}{rgb}{0.53, 0.81, 0.92}
\definecolor{Dandelion}{rgb}{0.94, 0.88, 0.19}
\definecolor{ForestGreen}{rgb}{0.13, 0.55, 0.13}
\definecolor{CornflowerBlue}{rgb}{0.6, 0.81, 0.93}
\definecolor{YellowOrange}{rgb}{1.0, 0.8, 0.0}%tangerineyellow
\definecolor{WildStrawberry}{rgb}{1.0, 0.26, 0.64}
\definecolor{Goldenrod}{rgb}{0.85, 0.65, 0.13}
\definecolor{BlueViolet}{rgb}{0.54, 0.17, 0.89}
\definecolor{RedOrange}{rgb}{0.8, 0.33, 0.0} %Burnt orange
\definecolor{OliveGreen}{rgb}{0.33, 0.42, 0.18}




\newcommand\tikzmark[2][]{
  \tikz[remember picture,inner sep=\tabcolsep,outer sep=0,baseline=(#1.base),align=left]{\node[minimum width=\hsize](#1){$#2$};}
}


\renewbibmacro*{index:name}[5]{%
  \usebibmacro{index:entry}{#1}
    {\iffieldundef{usera}{}{\thefield{usera}\actualoperator}\mkbibindexname{#2}{#3}{#4}{#5}}}

% \newcommand{\noop}[1]{}

\newcommand{\alert}[1]{#1}

\newcommand{\whitecell}{} 
\newcommand{\whitecellpp}[1]{%
  \parbox{1cm}{
%     \tikzmark[#1]{\raisebox{-1ex}{–}\raisebox{1ex}{\hspace{1ex}\textsubscript{\hai{PP}}}}
    {\textsubscript{PP}}
  }
}


\newcommand{\tablevspace}{\\[-.5em]}


\newcommand{\lightgraycell}{\cellcolor{gray!70!white}}
\newcommand{\lightgraycellpp}{\lightgraycell\textsubscript{\hai{PP}}}

\newcommand{\midgraycell}{\cellcolor{gray!30!white}}
\newcommand{\midgraycellpp}{\midgraycell\textsubscript{\hai{PP}}}

\newcommand{\darkgraycell}{\cellcolor{gray}}
\newcommand{\darkgraycellpp}{\darkgraycell\textsubscript{\hai{PP}}}

\newcommand{\blackcell}{\cellcolor{black}}

%Sprachen
\newcommand{\AHD}{\il{althochdeutsch}Ahd.} %LS hinzugefügt hoffentlich kein Konflikt zu {\ahd}was den gleichen il-Eintrag hat (soll im Index auch zusammenfallen)
\newcommand{\MHD}{\il{mittelhochdeutsch}Mhd.} %LS  wie  AHD möglicher Konflikt zu {\mhd}
\newcommand{\westgerm}{\il{westgermanisch}westgerm.} %LS hinzugefügt; kann gerne auch unter il{germanisch} indexiert werden
\newcommand{\hochaleman}{\il{hochalemannisch}hochaleman.}%LS hinzugefügt; kann gerne auch unter il{alemannisch} indexiert werden


\newcommand{\aj}{\il{altjiddisch}aj.}
\newcommand{\FiJi}{\il{Filmjiddisch}FiJi}
\newcommand{\LiJi}{\il{Literaturjiddisch}LiJi}
\newcommand{\LiJieins}{\il{Literaturjiddisch des 18. und 19. Jahrhunderts}LiJi1}
\newcommand{\LiJizwei}{\il{Literaturjiddisch des späten 20. und 21. Jahrhunderts}LiJi2}
\newcommand{\LiHe}{\il{Literaturhebräisch}LiHe}
\newcommand{\mj}{\il{mitteljiddisch}mj.}
\newcommand{\NOJ}{\il{Nordostjiddisch}NOJ}
\newcommand{\NÜJ}{\il{nördliches Übergangsjiddisch}NÜJ}
\newcommand{\NWJ}{\il{Nordwestjiddisch}NWJ}
\newcommand{\oj}{\il{ostjiddisch}oj.}
\newcommand{\OJ}{\il{Ostjiddisch}OJ}
\newcommand{\SOJ}{\il{Südostjiddisch}SOJ}
\newcommand{\SÜJ}{\il{südliches Übergangsjiddisch}SÜJ}
\newcommand{\SWJ}{\il{Südwestjiddisch}SWJ}
\newcommand{\urj}{\il{urjiddisch, protojiddisch}urj.}  
\newcommand{\wj}{\il{westjiddisch}wj.}
\newcommand{\WJ}{\il{Westjiddisch}WJ}
\newcommand{\ZOJ}{\il{Zentralostjiddisch}ZOJ}
\newcommand{\ZWJ}{\il{Zentralwestjiddisch}ZWJ} 
\newcommand{\afr}{\il{afrikaans}afr.}
\newcommand{\ahd}{\il{althochdeutsch}ahd.}
\newcommand{\aleman}{\il{alemannisch}aleman.}
\newcommand{\bair}{\il{bairisch}bair.}
\newcommand{\dän}{\il{dänisch}dän.}
\newcommand{\elsäss}{\il{elsässisches Niederalemannisch}elsäss}
\newcommand{\engl}{\il{englisch}engl.}
\newcommand{\frnhd}{\il{frühneuhochdeutsch}frnhd.}
\newcommand{\germ}{\il{germanisch}germ.}
\newcommand{\isl}{\il{isländisch}isl.}
\newcommand{\mhd}{\il{mittelhochdeutsch}mhd.}
\newcommand{\moselfränk}{\il{moselfränkisch}moselfränk.}
\newcommand{\ndt}{\il{niederdeutsch}ndt.}
\newcommand{\ndl}{\il{niederländisch}ndl.}
\newcommand{\rheinfränk}{\il{rheinfränkisch}rheinfränk.}%rs n fehlt
\newcommand{\schwäb}{\il{schwäbisch}schwäb.}
%\newcommand{\schwed}{\il{schwedisch}schwed.} %LS nicht mehr im Text
\newcommand{\westfl}{\il{westflämisch}westfl.}
\newcommand{\westfr}{\il{westfriesisch}westfr.} 
\newcommand{\fr}{\il{französisch}fr.}
\newcommand{\hebr}{\il{hebräisch}hebr.}
\newcommand{\itAL}{\il{italienisch}it.}  %it geht nicht weil schon besetzt
\newcommand{\poln}{\il{polnisch}poln.}
%\newcommand{\russ}{\il{russisch}russ.} %LS nicht mehr im Text
\newcommand{\tsch}{\il{tschechisch}tsch.} 
% Sachregister
\newcommand{\ACI}{\is{Accusativus cum infinitivo}ACI}
\newcommand{\Adv}{\is{Adverb}Adv.}
%\newcommand{\Adj}{\is{Adjektiv}Adj.}%LS nicht mehr im Text
\newcommand{\AdvP}{\is{Adverbialphrase}AdvP}%rs kein Spatium
\newcommand{\Akk}{\is{Akkusativ}Akk.}
\newcommand{\AP}{\is{Adjektivphrase}AP}
\newcommand{\Art}{\is{Artikel}Art.}
\newcommand{\CxG}{\is{ Construction grammar (Konstruktionsgrammatik)}CxG}
\newcommand{\Dat}{\is{Dativ}Dat.}
\newcommand{\f}{\is{feminin}f.}
\newcommand{\FK}{\is{Frequenzklasse}FK}
\newcommand{\Gen}{\is{Genitiv}Gen.}
%\newcommand{\intrans}{\is{intransitiv}intrans.}%LS nicht mehr im Text
\newcommand{\IPP}{\is{Infinitivus pro participio (Ersatzinfinitiv)}IPP}
%\newcommand{\Konj}{\is{Konjunktion}Konj.} %LS nicht mehr im Text
\newcommand{\LSK}{\is{linke Satzklammer}LSK}
% \newcommand{\m}{\is{maskulin}m.} % \m geht nicht, schon vergeben
\newcommand{\mask}{\is{maskulin}m.}
\newcommand{\MF}{\is{Mittelfeld}MF}
\newcommand{\NF}{\is{Nachfeld}NF} %LS neu hinzugefügt
\newcommand{\n}{\is{neutrum}n.}
\newcommand{\Nom}{\is{Nominativ}Nom.}
\newcommand{\NP}{\is{Nominalphrase}NP}
\newcommand{\OV}{\is{Objekt-Verb Grundwortstellung}OV}
\newcommand{\Pl}{\is{Plural}Pl.}
\newcommand{\PP}{\is{Pr\"apositionalphrase}PP}
\newcommand{\PPI}{\is{Participium pro infinitivo}PPI}
%\newcommand{\Präp}{\is{Präposition}Präp.}%LS nicht mehr im Text
\newcommand{\Pron}{\is{Pronomen}Pron.}
\newcommand{\RSK}{\is{rechte Satzklammer}RSK}
\newcommand{\Sg}{\is{Singular}Sg.}
\newcommand{\Stabw}{\is{Standardabweichung}$\sigma$}
% \newcommand{\trans}{\is{transitiv}trans.} % trans geht nicht, schon vergeben
\newcommand{\Dim}{\is{Diminutiv}Dim.} %LS neu hinzugefügt
\newcommand{\transtv}{\is{transitiv}trans.} 
\newcommand{\VO}{\is{Verb-Objekt Grundwortstellung}VO}
\newcommand{\VR}{\is{Verb raising}VR}
\newcommand{\VPR}{\is{Verb projection raising}VPR}
\newcommand{\Zsf}{\is{Zusammenfall}Zsf.}

%%% table in Appendix. Thanks to David Carlisle
%%% See http://tex.stackexchange.com/questions/343332/extra-wide-and-long-tables-in-book-publications

\newbox\xlhead
\newbox\xrhead
\def\cleartoevenside{%
	\clearpage
	\ifodd\value{page}\mbox{}\clearpage\fi
}
%\showoutput

\def\xhead#1{%
	\cleartoevenside
	\setbox0\vbox{}%
	\setbox2\vbox{}%
	\xrow{#1}%
	\setbox\xlhead\vbox{\hrule\vskip2pt\box0\vskip2pt\hrule\vskip2pt}%
	\setbox\xrhead\vbox{\hrule\vskip2pt\box2\vskip2pt\hrule\vskip2pt}%
	\setbox0\vbox{\stepcounter{table}\captionof*{table}{Tabelle \thetable: Phänomene im \hai{chrLiJi1}}\vskip\baselineskip\hrule\vskip2pt\copy\xlhead}%
	\setbox2\vbox{\captionof*{table}{Tabelle \thetable: Phänomene im \hai{chrLiJi1} (Fortsetzung)}\vskip\baselineskip\hrule\vskip2pt\copy\xrhead}%
}
\newcount\colcount
\def\xrow#1{%
	\setbox4\hbox{}%
	\setbox6\hbox{}%
	\colcount=0 \xxrow#1&%
	\setbox0\vbox{\unvbox0\box4}%
	\setbox2\vbox{\unvbox2\box6}%
	\ifdim\ht0>.95\textheight
	\box0\vfill
	\pagebreak
	\box2\vfill
	\pagebreak
	\setbox0\vbox{\captionof*{table}{Tabelle \thetable: Phänomene im \hai{chrLiJi1} (Fortsetzung)}\vskip\baselineskip\copy\xlhead}%
	\setbox2\vbox{\captionof*{table}{Tabelle \thetable: Phänomene im \hai{chrLiJi1} (Fortsetzung)}\vskip\baselineskip\copy\xrhead}%
	\fi}

\def\xfinish{%
	\box0\hrule\vskip2pt\hrule\vskip2pt\vfill
	\pagebreak
	\box2\hrule\vskip2pt\hrule\vskip2pt\vfill
	\pagebreak
}
\def\xxrow#1&{%
	\advance\colcount 1 %
	\setbox\ifnum\colcount<26 4\else 6\fi\hbox{%
		\unhbox\ifnum\colcount<26 4\else 6\fi
		\xdoformat{\strut\ignorespaces#1\ifhmode\unskip\fi}}%
	\ifnum\colcount<53 \expandafter\xxrow\fi}

\def\xdoformat{%
	\csname xformat%
	\expandafter\ifx\csname xformat\the\colcount\endcsname\relax
	*\else\the\colcount\fi
	\endcsname
}

\def\xdeclareformat#1#2{%
	\expandafter\def\csname xformat#1\endcsname##1{#2}}

\xdeclareformat{*}{\makebox[.5cm][c]{\scshape#1}}
\xdeclareformat{1}{\makebox[2cm][l]{\strut#1}}




\renewcommand\lsLanguageIndexTitle{Sprachregister}
\renewcommand\lsSubjectIndexTitle{Sachregister}
\renewcommand\lsNameIndexTitle{Personenregister}
% \renewcommand\lsTableOfContents{Personenregister}